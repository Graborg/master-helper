\section{Bakgrund}

Det senaste deciennets teknikutveckling har ökat behovet av ingenjörer drastiskt, samtidigt utbildas det fler ingenjörer än någonsin vid LTH \cite{lth}.
Titeln ingenjör kräver att vissa kriterier är uppfyllda, dessa kriterier är satta av högskoleverket \cite{?} och innefattar bland annat ett visst antal högskolepoäng inom en specialisering och ett antal högskolepoäng i avancerade kurser, men lämnar mycket till ingenjörsstudenten att välja själv.
Med sådan valfrihet kommer även oftast en förvirring angående vilka kurser som bör väljas men också om en ingenjörsexamen kan utfördas med det aktuella kursvalet.

I nuläget finns inget officiellt hjälpmedel för studenterna för att välja sina kurser. Istället förlitar sig universitetet på att studenten själv hanterar detta på egen hand.

\section{Syfte}

Projektet syftar till att ta reda på om studenterna tycker att det finns ett behov av ett verktyg för att planera sina valfria kurser, hur ett sådant verktyg bör se ut och hur det bör fungera och vad som bör prioriteras. Tillsist syftar projektet till att implementera en prototyp av ett sådant verktyg, med bas från undesökningarna och förbättras med hjälp av användartester.

\section{metod}

\subsection{Fokusgrupp}

Först och främst användes en gruppintervju av typen fokusgrupp, beskrivet så som beskrivet av Victoria Wibeck.\cite{wibeck} Intervjun gick ut på att moderera en grupp människor och låta dem diskutera fritt angående funktionalitet och gränssnitt med projektsyftet som kontext. Fokusgrupper valdes dels för att de är relativt tidsekonomiska dessutom kan de ge en bred bild som sedan kan uppföljas med en kvantitativ studie. \cite{wibeck} Wibeck beskriver även att 'Fokusgrupper' bör användas när det finns olikheter mellan deltagarna så att diskussionen drivs vidare. Fokusgruppen som användes var i visst avseende homogen, men från olika program. Dessutom bör alla deltagarna känna sig fria att uttrycka sina åsikter inför hela gruppen. Vilket inte alltid är självklart när det handlar om känsliga ämnen som diskuteras med t.ex. ansvarig personer. I gruppen fanns dock ingen som blev direkt berörd av åsikterna i gruppen och förhoppningsvis fanns alltså ingen risk för att någon skulle ta illa vid.
\subsection{Enketundersökning}

Efter den mer öppna diskussionen påbörjades en enkätundersökning för att ytterligare ta reda på åsikter om gränssnitt och funktionalitet och smalna in och konkretisera vad som bör implementeras i prototypen.

en enkät med ett brett under, efter det användes även
För att lägga en grund för använ

Först och främst användes en gruppintervju av typen fokusgrupp, beskrivet så som beskrivet av Victoria Wibeck.\cite{wibeck} Intervjun gick ut på att moderera en grupp människor och låta dem diskutera fritt angående funktionalitet och gränssnitt. Fokusgrupper valdes dels för att de är relativt tidsekonomiska dessutom kan de ge en bred bild som sedan kan uppföljas med en kvantitativ studie. \cite{wibeck} Wibeck beskriver även att Fokusgrupper bör användas när det finns olikheter mellan deltagarna så att diskussionen drivs vidare. Fokusgruppen som användes var i stor del homogen, men från olika program. Dessutom bör alla deltagarna känna sig fria att uttrycka sina åsikter inför de andra f

en enkät med ett brett under, efter det användes även
